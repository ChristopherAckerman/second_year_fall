\documentclass[dvipsnames]{article}
\usepackage{amsmath,amsthm,amssymb}
\usepackage{graphicx}
\usepackage{hyperref}
\usepackage{textcomp}
% \usepackage{dsfont}
\usepackage{tabularx}
\usepackage{tikz}
\usepackage{physics}
\usepackage{changepage}% http://ctan.org/pkg/changepage
\usetikzlibrary{scopes,calc,arrows}
\usepackage{setspace}
\usepackage[makeroom]{cancel}
\usepackage{enumitem}
\usepackage[margin=1in]{geometry}
\usepackage[T1]{fontenc}
\usepackage[utf8]{inputenc}
\usepackage{tabularx,ragged2e,booktabs,caption}
\usepackage{wrapfig,lipsum,booktabs}
\usepackage{hanging}
\usepackage{multicol}
\usepackage{multirow}
\usepackage{blindtext}
\usepackage{booktabs}
\usepackage{color}
\usepackage{dcolumn}
% \usepackage{minted}
% \definecolor{light}{rgb}{0.35, 0.35, 0.35}
% \def\light#1{{\color{light}#1}}
\usepackage{mathtools}
\newcommand{\p}{\mathbb{P}}
\newcommand{\E}{\mathbb{E}}
\newcommand{\R}{\mathbb{R}}
\newcommand{\Var}{\operatorname{Var}}
\newcommand{\gr}{\textcolor{ForestGreen}}
\newcommand{\rd}{\textcolor{red}}



\begin{document}
\section{Demand}
\begin{itemize}
\item Uses AIDS (Deaton and Muellbauer; should read at some point)
\item Two-step nested choice model; doesn't impose any restrictions on cross-price elasticties.
\item Estimating Equations:
  \begin{equation}
    \omega_{Grt} = \alpha_G + \alpha_{Gr} + \sum_H \gamma_{GH} \ln P_{Hrt} + \beta_G \ln \left(\frac{X_{rt}}{P_{rt}}\right) + \varepsilon_{Grt} \tag{Upper Stage}
  \end{equation}
  At the lower level, we have
  \begin{equation*}
  \hspace{-8em}  \omega_{irt} = \alpha_i + \alpha_{ir} + \underbrace{\gamma_{ii} \ln  p_{irt}}_{\text{own price}} + \underbrace{\gamma_{i, 10} \ln p_{jrt, j = D_i^{10}}}_{\text{same molecule; different country}} + \underbrace{\sum_{j \in D_i^{01}}[\gamma_{i, 01} \ln p_{jrt}]}_{\text{different molecule; same country}} + \underbrace{\sum_{j \in D_i^{00}}[\gamma_{i, 00} \ln p_{jrt}]}_{\text{different molecule, different country}} + \beta_i \ln \left(\frac{X_{Qrt}}{P_{Qrt}}\right) + \varepsilon_{irt} %\tag{Lower Level}
  \end{equation*}
\item The authors estimate this with IV and OLS; IV is the right way to do it but they also confirm that OLS doesn't give \emph{crazy} results.
\item \rd{Add table of instruments and estimation}
  \item These parameters, somehow (due to AIDS?) give us the elasticity of demand \rd{Check the AIDS algebra for this and check with classmates}
\end{itemize}

\section{Supply}
\begin{itemize}
\item In order to do anything interesting, we need to know MC but we don't observe it.
\item The usual approach in IO is to exploit firms' equilibrium conditions; general sketch:
  \begin{enumerate}
  \item Assume something about $MC_i$ for firms $i$
  \item Assume something about the competition structure (oligopoly market; compete via Bertrand)
  \item Assume something about firm behavior (period-by-period profit maximization)
    \item Derive first-order-conditions for firms
  \end{enumerate}
  \item The authors can't do this, even as a constrained optimization problem, because the existing price regulations make this too difficult. \rd{Why?}
  \item Instead the authors do an exercise to bound marginal prices; two extreme assumptions for an upper and lower bound:
    \begin{enumerate}
    \item Perfect competition $\implies$ highest marginal cost $\implies c_i^U = p_i$
      \item Perfect collusion $\approx$ monopoly $\implies$ lowest marginal cost $\implies c_i^L = p_i \cdot \left(1 + \frac{1}{\varepsilon_{ii}(p_i p_j)}\right)$
    \end{enumerate}
    \item Most of the exercises in the paper use the lower bound, since this represents the largest potential losses for firms under TRIPS (most to lose from patent enforcement)
\end{itemize}
\end{document}

%%% Local Variables:
%%% mode: latex
%%% TeX-master: t
%%% End:
