\documentclass{beamer}
\input{../slides_preamble}

\begin{document}
\begin{frame}{Research Question}
  \begin{itemize}
  \item  Effects of enforcing existing regulation in India
    \vfill
  \item Currently a bunch of firms flout international patent law; what happens if we enforce the law?
    \vfill
  \item Reduced competition: Patent protection gives firms market power
    \vfill
  \item What will the firms do with this market power?
    \vfill
    \item Will it hurt consumers?
  \end{itemize}
\end{frame}
%
\begin{frame}{Approach}
  \begin{itemize}
  \item Estimate Demand system for Quinalones (class of drugs) in India; need to estimate demand elasticities for this
    \vfill
  \item Product approach: define drugs along two dimensions
    \begin{enumerate}
    \item Molecule
      \item Producer nationality (foreign vs. domestic)
    \end{enumerate}
    \vfill
  \item Bound counterfactuals with two extreme cases
    \begin{enumerate}
    \item Perfect competition; $c_i^U = p_i$
      \item Perfect collusion; $c_i^L = p_i \cdot \left(1 + \frac{1}{\varepsilon_{ii}(p_i p_j)}\right) $
    \end{enumerate}
  \end{itemize}
\end{frame}
%
\begin{frame}{Demand (Upper Stage)}
  \begin{itemize}
  \item AIDS model of demand; two-step choice
    \vfill
  \item In the first stage, allocate spending across categories of drugs:
    \begin{equation}
      \label{eq:1}
    \hspace{-4em} \omega_{Grt} = \alpha_G + \alpha_{Gr} + \sum_H \gamma_{GH} \ln P_{Hrt} + \beta_G \ln \left(\frac{X_{rt}}{P_{rt}}\right) + \varepsilon_{Grt} \tag{Upper Stage}
    \end{equation}
    \vfill
    \item Authors include $r$ subscripts to allow different prices in each region 
  \end{itemize}
\end{frame}
%
\begin{frame}{Demand (Lower Level)}
  \begin{itemize}
  \item In the second step, consumers choose a product, and each product is defined by its molecule and its manufacturer (foreign vs domestic.
  \end{itemize}
  \begin{align*}
    \omega_{irt} &= \alpha_i + \alpha_{ir} \\
                 &\ + \underbrace{\gamma_{ii} \ln  p_{irt}}_{\text{own price}} + \overbrace{\gamma_{i, 10} \ln p_{jrt, j = D_i^{10}}}^{\text{same molecule; different country}}\\
                 &\ + \underbrace{\sum_{j \in D_i^{01}}[\gamma_{i, 01} \ln p_{jrt}]}_{\text{different molecule; same country}} + \underbrace{\sum_{j \in D_i^{00}}[\gamma_{i, 00} \ln p_{jrt}]}_{\text{different molecule, different country}}\\
    &\ + \beta_i \ln \left(\frac{X_{Qrt}}{P_{Qrt}}\right) + \varepsilon_{irt} %\tag{Lower Level}
  \end{align*}
\end{frame}
%
\begin{frame}{Estimation}
  \begin{itemize}
  \item The authors can't use OLS
    \begin{enumerate}
    \item \emph{Not} worried about endogeneity (because price controls screw up the firm's FOCs anyway)
      \item \emph{Are} worried about measurement error
    \end{enumerate}
    \vfill
  \item Construct an ``approximate'' price index (with measurement error)
    \vfill
  \item Instruments must be
    \begin{enumerate}
    \item Correlated with the true price index, $p_j^A$
      \item Uncorrelated with error term $\varepsilon_i$
    \end{enumerate}
  \end{itemize}
\end{frame}
%
\begin{frame}{Instruments}
  \begin{itemize}
  \item Authors use SKU prices as instruments for the price index
    \vfill
  \item I'm not sure if I believe them\ldots They even mention that, if new SKUs indicate improving quality, etc. then their instruments might be correlated with the error term (thoughts?)
    \vfill
  \item List of instruments:
    \begin{enumerate}
    \item Number of SKUs in each group
    \item Prices of the five largest SKUs in each group
      \item All exogenous variables
    \end{enumerate}
  \end{itemize}
\end{frame}
%
\begin{frame}{Counterfactual}
  \begin{itemize}
  \item Remove different numbers of domestic drugs
    \vfill
  \item Assume the foreign firm(s) have monopoly power
    \begin{itemize}
    \item Either full monopolists, or facing price constraints, depending on scenario
    \end{itemize}
    \vfill
  \item Calculate the price the new monopolist would charge (potentially subject to constraints)
    \vfill
    \item Welfare analysis
  \end{itemize}
\end{frame}
%
\begin{frame}{Findings}
  \begin{itemize}
  \item Full model helps with counterfactual substitution patterns
    \vfill
  \item ``Welfare'' is Consumer Surplus $+$ Firm Profits
    \vfill
  \item Welfare changes measured as Compensating Variation
    \vfill
  \item Three driving forces
    \begin{enumerate}
    \item Lost variety
    \item Expenditure switching
      \item Reduced competition
    \end{enumerate}
  \end{itemize}
\end{frame}
%
\begin{frame}{Findings}
  \begin{itemize}
  \item  Domestic products are very good substitutes for each other
    \vfill
  \item Most of the welfare changes ($>80\%$) are due to lost consumer welfare
    \vfill
  \item Authors claim that this is likely due to lost ``variety''
    \begin{itemize}
    \item Domestic products have more different types (dosage/pill, pills/bottle, etc.)
      \item Domestic products also have better distribution networks; more readily available at local pharmacies
    \end{itemize}
  \end{itemize}
\end{frame}
%
\begin{frame}{Critique}
  \begin{itemize}
  \item The competition on the supply side seems a bit suspect
    \vfill
  \item If ``variety'' or ``distribution network'' is an important aspect of consumer choice, those product attributes should be in the model somewhere
    \vfill
  \item I don't think that ``foreign vs. domestic'' does a good enough job capturing this
  \end{itemize}
\end{frame}
%
\begin{frame}{Proposed Counterfactual}
  \begin{itemize}
  \item Firms have two choice variables:
    \begin{enumerate}
    \item Price
      \item Quality
    \end{enumerate}
    \vfill
  \item In addition to a $p_i$ estimate, firms have another parameter in their decision, $v_i$
    \begin{itemize}
    \item This is the price of producing ``variety''; think of it as the cost of establishing a distribution network
      \begin{itemize}
      \item (Naturally domestic firms would be able to establish a widespread distribution network more cheaply)
      \end{itemize}
    \end{itemize}
    \vfill
    \item Firms now choose a pair $\{p_i, N_i\}$, where $N_i$ is the size of their distribution network, produced at a cost of $N(v_i)$
  \end{itemize}
\end{frame}
%
\begin{frame}{What might happen?}
  \begin{itemize}
  \item Given the existing competition, foreign firms might not find it profitable to invest in their (expensive) distribution network
    \vfill
  \item However, given the regulation, they may then find it profitable to invest in a large distribution network, potentially as large or larger than the existing domestic networks (economies of scale)
    \vfill
  \item This dynamic supply-side response will restore a lot of the ``variety'' driving the welfare losses in the original counterfactual
    \vfill
    \item I saw this dynamic supply-side model in \emph{Neilson, Allende and Gallego} (2019)
  \end{itemize}
\end{frame}
\end{document}

%%% Local Variables:
%%% mode: latex
%%% TeX-master: t
%%% End:
