\documentclass[dvipsnames]{article}
\usepackage{minted}
\documentclass[dvipsnames]{article}
\usepackage{amsmath,amsthm,amssymb}
\usepackage{graphicx}
\usepackage{hyperref}
\usepackage{textcomp}
% \usepackage{dsfont}
\usepackage{tabularx}
\usepackage{tikz}
\usepackage{physics}
\usepackage{changepage}% http://ctan.org/pkg/changepage
\usetikzlibrary{scopes,calc,arrows}
\usepackage{setspace}
\usepackage[makeroom]{cancel}
\usepackage{enumitem}
\usepackage[margin=1in]{geometry}
\usepackage[T1]{fontenc}
\usepackage[utf8]{inputenc}
\usepackage{tabularx,ragged2e,booktabs,caption}
\usepackage{wrapfig,lipsum,booktabs}
\usepackage{hanging}
\usepackage{multicol}
\usepackage{multirow}
\usepackage{blindtext}
\usepackage{booktabs}
\usepackage{color}
\usepackage{dcolumn}
% \usepackage{minted}
% \definecolor{light}{rgb}{0.35, 0.35, 0.35}
% \def\light#1{{\color{light}#1}}
\usepackage{mathtools}
\newcommand{\p}{\mathbb{P}}
\newcommand{\E}{\mathbb{E}}
\newcommand{\R}{\mathbb{R}}
\newcommand{\Var}{\operatorname{Var}}
\newcommand{\gr}{\textcolor{ForestGreen}}
\newcommand{\rd}{\textcolor{red}}


\title{IO Problem Set 1 (BLP)}
\author{Chris Ackerman\thanks{I worked on this problem set with Luna Shen, David Kerns, and Benedikt Graf}}
\date{\today}


\begin{document}
\maketitle
\section*{Problem 1}
\subsection*{Estimate the Model using OLS, with price and promotion as characteristics}

\begin{minted}{python}
  #1.1: Estimate using OLS with price and promotion as product characteristics.
res_log1 = smf.ols('Y ~ prices + prom_', data=otc_dataDf).fit()
\end{minted}
\begin{table}[htp]
  \centering
  \begin{center}
\begin{tabular}{lclc}
\toprule
\textbf{Dep. Variable:}    &        Y         & \textbf{  R-squared:         } &     0.158   \\
\textbf{Model:}            &       OLS        & \textbf{  Adj. R-squared:    } &     0.158   \\
\textbf{Method:}           &  Least Squares   & \textbf{  F-statistic:       } &     3610.   \\
\textbf{Date:}             & Thu, 04 Nov 2021 & \textbf{  Prob (F-statistic):} &     0.00    \\
\textbf{Time:}             &     05:19:39     & \textbf{  Log-Likelihood:    } &   -56307.   \\
\textbf{No. Observations:} &       38544      & \textbf{  AIC:               } & 1.126e+05   \\
\textbf{Df Residuals:}     &       38541      & \textbf{  BIC:               } & 1.126e+05   \\
\textbf{Df Model:}         &           2      & \textbf{                     } &             \\
\textbf{Covariance Type:}  &    nonrobust     & \textbf{                     } &             \\
\bottomrule
\end{tabular}
\begin{tabular}{lcccccc}
                   & \textbf{coef} & \textbf{std err} & \textbf{t} & \textbf{P$> |$t$|$} & \textbf{[0.025} & \textbf{0.975]}  \\
\midrule
\textbf{Intercept} &      -7.6267  &        0.014     &  -532.839  &         0.000        &       -7.655    &       -7.599     \\
\textbf{prices}    &      -0.2496  &        0.003     &   -84.768  &         0.000        &       -0.255    &       -0.244     \\
\textbf{prom\_}    &      -0.0311  &        0.019     &    -1.653  &         0.098        &       -0.068    &        0.006     \\
\bottomrule
\end{tabular}
\begin{tabular}{lclc}
\textbf{Omnibus:}       & 1648.591 & \textbf{  Durbin-Watson:     } &    0.434  \\
\textbf{Prob(Omnibus):} &   0.000  & \textbf{  Jarque-Bera (JB):  } & 1461.418  \\
\textbf{Skew:}          &  -0.415  & \textbf{  Prob(JB):          } &     0.00  \\
\textbf{Kurtosis:}      &   2.529  & \textbf{  Cond. No.          } &     17.7  \\
\bottomrule
\end{tabular}
%\caption{OLS Regression Results}
\end{center}

Notes: \newline
 [1] Standard Errors assume that the covariance matrix of the errors is correctly specified.
%%% Local Variables:
%%% mode: latex
%%% TeX-master: t
%%% End:

\end{table}
\newpage

\subsection*{Estimate the Model using OLS, with price and promotion as characteristics, and brand dummies}

\begin{minted}{python}
#1.2: Estimate using OLS with price and promotion as product characteristics and brand dummies.
res_log2 = smf.ols('Y ~ prices + prom_ + C(brand)', data=otc_dataDf).fit()
  \end{minted}
\begin{table}[htp]
  \centering
  
\begin{center}
\begin{tabular}{lclc}
\toprule
\textbf{Dep. Variable:}    &        Y         & \textbf{  R-squared:         } &     0.654   \\
\textbf{Model:}            &       OLS        & \textbf{  Adj. R-squared:    } &     0.654   \\
\textbf{Method:}           &  Least Squares   & \textbf{  F-statistic:       } &     6081.   \\
\textbf{Date:}             & Thu, 04 Nov 2021 & \textbf{  Prob (F-statistic):} &     0.00    \\
\textbf{Time:}             &     05:20:51     & \textbf{  Log-Likelihood:    } &   -39138.   \\
\textbf{No. Observations:} &       38544      & \textbf{  AIC:               } & 7.830e+04   \\
\textbf{Df Residuals:}     &       38531      & \textbf{  BIC:               } & 7.841e+04   \\
\textbf{Df Model:}         &          12      & \textbf{                     } &             \\
\textbf{Covariance Type:}  &    nonrobust     & \textbf{                     } &             \\
\bottomrule
\end{tabular}
\begin{tabular}{lcccccc}
                        & \textbf{coef} & \textbf{std err} & \textbf{t} & \textbf{P$> |$t$|$} & \textbf{[0.025} & \textbf{0.975]}  \\
\midrule
\textbf{Intercept}      &      -6.0745  &        0.036     &  -167.065  &         0.000        &       -6.146    &       -6.003     \\
\textbf{C(brand)[T.2]}  &      -0.0048  &        0.022     &    -0.218  &         0.828        &       -0.048    &        0.039     \\
\textbf{C(brand)[T.3]}  &      -0.4578  &        0.040     &   -11.502  &         0.000        &       -0.536    &       -0.380     \\
\textbf{C(brand)[T.4]}  &      -0.4303  &        0.017     &   -25.951  &         0.000        &       -0.463    &       -0.398     \\
\textbf{C(brand)[T.5]}  &      -0.8868  &        0.024     &   -37.403  &         0.000        &       -0.933    &       -0.840     \\
\textbf{C(brand)[T.6]}  &      -1.3850  &        0.051     &   -27.408  &         0.000        &       -1.484    &       -1.286     \\
\textbf{C(brand)[T.7]}  &      -1.6527  &        0.018     &   -94.139  &         0.000        &       -1.687    &       -1.618     \\
\textbf{C(brand)[T.8]}  &      -2.2856  &        0.016     &  -141.034  &         0.000        &       -2.317    &       -2.254     \\
\textbf{C(brand)[T.9]}  &      -1.9340  &        0.017     &  -111.950  &         0.000        &       -1.968    &       -1.900     \\
\textbf{C(brand)[T.10]} &      -1.8983  &        0.022     &   -87.306  &         0.000        &       -1.941    &       -1.856     \\
\textbf{C(brand)[T.11]} &      -2.1754  &        0.019     &  -113.355  &         0.000        &       -2.213    &       -2.138     \\
\textbf{prices}         &      -0.3412  &        0.010     &   -33.864  &         0.000        &       -0.361    &       -0.321     \\
\textbf{prom\_}         &       0.3294  &        0.013     &    26.122  &         0.000        &        0.305    &        0.354     \\
\bottomrule
\end{tabular}
\begin{tabular}{lclc}
\textbf{Omnibus:}       & 2773.498 & \textbf{  Durbin-Watson:     } &    1.024  \\
\textbf{Prob(Omnibus):} &   0.000  & \textbf{  Jarque-Bera (JB):  } & 3867.305  \\
\textbf{Skew:}          &  -0.618  & \textbf{  Prob(JB):          } &     0.00  \\
\textbf{Kurtosis:}      &   3.938  & \textbf{  Cond. No.          } &     112.  \\
\bottomrule
\end{tabular}
%\caption{OLS Regression Results}
\end{center}

Notes: \newline
 [1] Standard Errors assume that the covariance matrix of the errors is correctly specified.

  \end{table}
  \newpage

\subsection*{Estimate the Model using OLS, with price and promotion as characteristics, and store-brand dummies}
\begin{minted}{python}
#1.3. Using OLS with price and promotion as product characteristics and store-brand 
#(the interaction of brand and store) dummies.

res_log3 = smf.ols('Y ~ prices + prom_ + C(brand)*C(store)', data=otc_dataDf).fit()
  \end{minted}
\begin{table}[htp]
  \centering
  \begin{center}
\begin{tabular}{lclc}
\toprule
\textbf{Dep. Variable:}                 &        Y         & \textbf{  R-squared:         } &     0.722   \\
\textbf{Model:}                         &       OLS        & \textbf{  Adj. R-squared:    } &     0.716   \\
\textbf{Method:}                        &  Least Squares   & \textbf{  F-statistic:       } &     121.9   \\
\textbf{Date:}                          & Thu, 04 Nov 2021 & \textbf{  Prob (F-statistic):} &     0.00    \\
\textbf{Time:}                          &     05:21:32     & \textbf{  Log-Likelihood:    } &   -34946.   \\
\textbf{No. Observations:}              &       38544      & \textbf{  AIC:               } & 7.150e+04   \\
\textbf{Df Residuals:}                  &       37739      & \textbf{  BIC:               } & 7.839e+04   \\
\textbf{Df Model:}                      &         804      & \textbf{                     } &             \\
\textbf{Covariance Type:}               &    nonrobust     & \textbf{                     } &             \\
\bottomrule
\end{tabular}
\begin{tabular}{lcccccc}
                                        & \textbf{coef} & \textbf{std err} & \textbf{t} & \textbf{P$> |$t$|$} & \textbf{[0.025} & \textbf{0.975]}  \\
\midrule
\textbf{Intercept}                      &      -6.1994  &        0.093     &   -66.365  &         0.000        &       -6.382    &       -6.016     \\
\textbf{prices}                         &      -0.3302  &        0.010     &   -34.349  &         0.000        &       -0.349    &       -0.311     \\
\textbf{prom\_}                         &       0.3288  &        0.011     &    28.619  &         0.000        &        0.306    &        0.351     \\
\bottomrule
\end{tabular}
\begin{tabular}{lclc}
\textbf{Omnibus:}       & 4044.934 & \textbf{  Durbin-Watson:     } &    1.268  \\
\textbf{Prob(Omnibus):} &   0.000  & \textbf{  Jarque-Bera (JB):  } & 7222.052  \\
\textbf{Skew:}          &  -0.721  & \textbf{  Prob(JB):          } &     0.00  \\
\textbf{Kurtosis:}      &   4.555  & \textbf{  Cond. No.          } & 4.08e+03  \\
\bottomrule
\end{tabular}
%\caption{OLS Regression Results}
\end{center}

Notes: \newline
 [1] Standard Errors assume that the covariance matrix of the errors is correctly specified. \newline
 [2] The condition number is large, 4.08e+03. This might indicate that there are \newline
 strong multicollinearity or other numerical problems. Dummies omitted.
%%% Local Variables:
%%% mode: latex
%%% TeX-master: t
%%% End:

  \end{table}
  \newpage

\subsection*{Estimate the models from parts 1--3 using wholesale cost as an instrument}
\begin{minted}{python}
  # OLS with price and promotion as product characteristics, using wholesale cost as instrument
wholeSale_IV1 = iv.IV2SLS.from_formula('Y ~ 1 + [prices ~ cost_] + prom_ ', data=otc_dataDf).fit()
\end{minted}
\begin{table}[htp]
  \centering
  \begin{center}
\begin{tabular}{lclc}
\toprule
\textbf{Dep. Variable:}    &         Y          & \textbf{  R-squared:         } &      0.1531      \\
\textbf{Estimator:}        &      IV-2SLS       & \textbf{  Adj. R-squared:    } &      0.1531      \\
\textbf{No. Observations:} &       38544        & \textbf{  F-statistic:       } &      5255.4      \\
\textbf{Date:}             &  Thu, Nov 04 2021  & \textbf{  P-value (F-stat)   } &      0.0000      \\
\textbf{Time:}             &      05:25:20      & \textbf{  Distribution:      } &     chi2(2)      \\
\textbf{Cov. Estimator:}   &       robust       & \textbf{                     } &                  \\
\textbf{}                  &                    & \textbf{                     } &                  \\
\bottomrule
\end{tabular}
\begin{tabular}{lcccccc}
                   & \textbf{Parameter} & \textbf{Std. Err.} & \textbf{T-stat} & \textbf{P-value} & \textbf{Lower CI} & \textbf{Upper CI}  \\
\midrule
\textbf{Intercept} &      -7.8181       &       0.0150       &     -520.30     &      0.0000      &      -7.8476      &      -7.7887       \\
\textbf{prom\_}    &      -0.0068       &       0.0170       &     -0.4034     &      0.6866      &      -0.0401      &       0.0264       \\
\textbf{prices}    &      -0.2066       &       0.0029       &     -71.884     &      0.0000      &      -0.2122      &      -0.2009       \\
\bottomrule
\end{tabular}
%\caption{IV-2SLS Estimation Summary}
\end{center}

Endogenous: prices \newline
 Instruments: cost\_ \newline
 Robust Covariance (Heteroskedastic) \newline
 Debiased: False
%%% Local Variables:
%%% mode: latex
%%% TeX-master: t
%%% End:

  \end{table}
 
  \newpage

  \begin{minted}{python}
    # OLS with price and promotion as product characteristics and brand dummies
# using wholesale cost as instrument

wholeSale_IV2 = iv.IV2SLS.from_formula('Y ~ 1 + [prices ~ cost_] + prom_ + C(brand)', data=otc_dataDf).fit()
\end{minted}
\begin{table}[htp]
  \centering
  \begin{center}
\begin{tabular}{lclc}
\toprule
\textbf{Dep. Variable:}    &         Y          & \textbf{  R-squared:         } &      0.6446      \\
\textbf{Estimator:}        &      IV-2SLS       & \textbf{  Adj. R-squared:    } &      0.6445      \\
\textbf{No. Observations:} &       38544        & \textbf{  F-statistic:       } &     9.69e+04     \\
\textbf{Date:}             &  Thu, Nov 04 2021  & \textbf{  P-value (F-stat)   } &      0.0000      \\
\textbf{Time:}             &      05:26:08      & \textbf{  Distribution:      } &     chi2(12)     \\
\textbf{Cov. Estimator:}   &       robust       & \textbf{                     } &                  \\
\textbf{}                  &                    & \textbf{                     } &                  \\
\bottomrule
\end{tabular}
\begin{tabular}{lcccccc}
                        & \textbf{Parameter} & \textbf{Std. Err.} & \textbf{T-stat} & \textbf{P-value} & \textbf{Lower CI} & \textbf{Upper CI}  \\
\midrule
\textbf{Intercept}      &      -7.2171       &       0.0652       &     -110.69     &      0.0000      &      -7.3449      &      -7.0893       \\
\textbf{C(brand)[T.2]}  &      -0.5161       &       0.0309       &     -16.703     &      0.0000      &      -0.5766      &      -0.4555       \\
\textbf{C(brand)[T.3]}  &      -1.6627       &       0.0698       &     -23.833     &      0.0000      &      -1.7994      &      -1.5260       \\
\textbf{C(brand)[T.4]}  &      -0.2833       &       0.0147       &     -19.314     &      0.0000      &      -0.3121      &      -0.2546       \\
\textbf{C(brand)[T.5]}  &      -1.4660       &       0.0358       &     -40.904     &      0.0000      &      -1.5363      &      -1.3958       \\
\textbf{C(brand)[T.6]}  &      -2.9699       &       0.0914       &     -32.484     &      0.0000      &      -3.1491      &      -2.7907       \\
\textbf{C(brand)[T.7]}  &      -1.4158       &       0.0181       &     -78.185     &      0.0000      &      -1.4513      &      -1.3803       \\
\textbf{C(brand)[T.8]}  &      -2.3613       &       0.0137       &     -172.35     &      0.0000      &      -2.3882      &      -2.3344       \\
\textbf{C(brand)[T.9]}  &      -2.1365       &       0.0169       &     -126.55     &      0.0000      &      -2.1696      &      -2.1034       \\
\textbf{C(brand)[T.10]} &      -1.4120       &       0.0321       &     -44.036     &      0.0000      &      -1.4749      &      -1.3492       \\
\textbf{C(brand)[T.11]} &      -2.5260       &       0.0283       &     -89.396     &      0.0000      &      -2.5814      &      -2.4707       \\
\textbf{prom\_}         &       0.4307       &       0.0145       &      29.801     &      0.0000      &       0.4024      &       0.4590       \\
\textbf{prices}         &      -0.0081       &       0.0189       &     -0.4287     &      0.6682      &      -0.0452      &       0.0290       \\
\bottomrule
\end{tabular}
%\caption{IV-2SLS Estimation Summary}
\end{center}

Endogenous: prices \newline
 Instruments: cost\_ \newline
 Robust Covariance (Heteroskedastic) \newline
 Debiased: False
%%% Local Variables:
%%% mode: latex
%%% TeX-master: t
%%% End:

  \end{table}

  \begin{minted}{python}
# OLS with price and promotion as product characteristics and brand dummies
# using wholesale cost as instrument

wholeSale_IV3 = iv.IV2SLS.from_formula('Y ~ 1 + [prices ~ cost_] + prom_ + C(brand)*C(store)', data=otc_dataDf).fit()
    \end{minted}
\begin{table}[htp]
  \centering
  \begin{center}
\begin{tabular}{lclc}
\toprule
\textbf{Dep. Variable:}                 &         Y          & \textbf{  R-squared:         } &      0.7150      \\
\textbf{Estimator:}                     &      IV-2SLS       & \textbf{  Adj. R-squared:    } &      0.7090      \\
\textbf{No. Observations:}              &       38544        & \textbf{  F-statistic:       } &    1.766e+05     \\
\textbf{Date:}                          &  Thu, Nov 04 2021  & \textbf{  P-value (F-stat)   } &      0.0000      \\
\textbf{Time:}                          &      05:27:37      & \textbf{  Distribution:      } &    chi2(804)     \\
\textbf{Cov. Estimator:}                &       robust       & \textbf{                     } &                  \\
\textbf{}                               &                    & \textbf{                     } &                  \\
\bottomrule
\end{tabular}
\begin{tabular}{lcccccc}
                                        & \textbf{Parameter} & \textbf{Std. Err.} & \textbf{T-stat} & \textbf{P-value} & \textbf{Lower CI} & \textbf{Upper CI}  \\
\midrule
\textbf{Intercept}                      &      -7.2138       &       0.0775       &     -93.106     &      0.0000      &      -7.3657      &      -7.0620       \\
\textbf{prom\_}                         &       0.4193       &       0.0132       &      31.774     &      0.0000      &       0.3934      &       0.4451       \\
\textbf{prices}                         &      -0.0346       &       0.0178       &     -1.9442     &      0.0519      &      -0.0695      &       0.0003       \\
\bottomrule
\end{tabular}
\caption{IV-2SLS Estimation Summary, dummies suppressed }
\end{center}

Endogenous: prices \newline
 Instruments: cost\_ \newline
 Robust Covariance (Heteroskedastic) \newline
 Debiased: False
%%% Local Variables:
%%% mode: latex
%%% TeX-master: t
%%% End:

  \end{table}

\newpage
  \subsection*{Estimate the models from parts 1--3 using the Hausman instrument}

\begin{table}[htp]
  \centering
  \begin{center}
\begin{tabular}{lclc}
\toprule
\textbf{Dep. Variable:}    &         Y          & \textbf{  R-squared:         } &      0.1578      \\
\textbf{Estimator:}        &      IV-2SLS       & \textbf{  Adj. R-squared:    } &      0.1577      \\
\textbf{No. Observations:} &       38544        & \textbf{  F-statistic:       } &      9465.0      \\
\textbf{Date:}             &  Thu, Nov 04 2021  & \textbf{  P-value (F-stat)   } &      0.0000      \\
\textbf{Time:}             &      05:40:33      & \textbf{  Distribution:      } &     chi2(2)      \\
\textbf{Cov. Estimator:}   &       robust       & \textbf{                     } &                  \\
\textbf{}                  &                    & \textbf{                     } &                  \\
\bottomrule
\end{tabular}
\begin{tabular}{lcccccc}
                   & \textbf{Parameter} & \textbf{Std. Err.} & \textbf{T-stat} & \textbf{P-value} & \textbf{Lower CI} & \textbf{Upper CI}  \\
\midrule
\textbf{Intercept} &      -7.6143       &       0.0135       &     -565.64     &      0.0000      &      -7.6407      &      -7.5879       \\
\textbf{prom\_}    &      -0.0327       &       0.0170       &     -1.9257     &      0.0541      &      -0.0660      &       0.0006       \\
\textbf{prices}    &      -0.2524       &       0.0026       &     -97.062     &      0.0000      &      -0.2574      &      -0.2473       \\
\bottomrule
\end{tabular}
%\caption{IV-2SLS Estimation Summary}
\end{center}

Endogenous: prices \newline
 Instruments: pricestore1, pricestore2, pricestore3, pricestore4, pricestore5, pricestore6, pricestore7, pricestore8, pricestore9, pricestore10, pricestore11, pricestore12, pricestore13, pricestore14, pricestore15, pricestore16, pricestore17, pricestore18, pricestore19, pricestore20, pricestore21, pricestore22, pricestore23, pricestore24, pricestore25, pricestore26, pricestore27, pricestore28, pricestore29, pricestore30 \newline
 Robust Covariance (Heteroskedastic) \newline
 Debiased: False

%%% Local Variables:
%%% mode: latex
%%% TeX-master: t
%%% End:

  \end{table}
  \newpage

\begin{table}[htp]
  \centering
  \begin{center}
\begin{tabular}{lclc}
\toprule
\textbf{Dep. Variable:}        &         Y          & \textbf{  R-squared:         } &      0.6511      \\
\textbf{Estimator:}            &      IV-2SLS       & \textbf{  Adj. R-squared:    } &      0.6510      \\
\textbf{No. Observations:}     &       38544        & \textbf{  F-statistic:       } &    9.529e+04     \\
\textbf{Date:}                 &  Thu, Nov 04 2021  & \textbf{  P-value (F-stat)   } &      0.0000      \\
\textbf{Time:}                 &      05:41:01      & \textbf{  Distribution:      } &     chi2(12)     \\
\textbf{Cov. Estimator:}       &       robust       & \textbf{                     } &                  \\
\textbf{}                      &                    & \textbf{                     } &                  \\
\bottomrule
\end{tabular}
\begin{tabular}{lcccccc}
                               & \textbf{Parameter} & \textbf{Std. Err.} & \textbf{T-stat} & \textbf{P-value} & \textbf{Lower CI} & \textbf{Upper CI}  \\
\midrule
\textbf{Intercept}             &      -5.4061       &       0.0539       &     -100.33     &      0.0000      &      -5.5117      &      -5.3005       \\
\textbf{C(product\_ids)[T.2]}  &       0.2942       &       0.0259       &      11.370     &      0.0000      &       0.2435      &       0.3449       \\
\textbf{C(product\_ids)[T.3]}  &       0.2471       &       0.0574       &      4.3071     &      0.0000      &       0.1347      &       0.3595       \\
\textbf{C(product\_ids)[T.4]}  &      -0.5162       &       0.0140       &     -36.812     &      0.0000      &      -0.5437      &      -0.4888       \\
\textbf{C(product\_ids)[T.5]}  &      -0.5479       &       0.0298       &     -18.367     &      0.0000      &      -0.6064      &      -0.4894       \\
\textbf{C(product\_ids)[T.6]}  &      -0.4578       &       0.0751       &     -6.0967     &      0.0000      &      -0.6050      &      -0.3107       \\
\textbf{C(product\_ids)[T.7]}  &      -1.7913       &       0.0169       &     -105.81     &      0.0000      &      -1.8245      &      -1.7581       \\
\textbf{C(product\_ids)[T.8]}  &      -2.2413       &       0.0143       &     -156.43     &      0.0000      &      -2.2694      &      -2.2132       \\
\textbf{C(product\_ids)[T.9]}  &      -1.8155       &       0.0156       &     -116.73     &      0.0000      &      -1.8460      &      -1.7850       \\
\textbf{C(product\_ids)[T.10]} &      -2.1827       &       0.0280       &     -77.963     &      0.0000      &      -2.2376      &      -2.1279       \\
\textbf{C(product\_ids)[T.11]} &      -1.9703       &       0.0231       &     -85.436     &      0.0000      &      -2.0155      &      -1.9251       \\
\textbf{prom\_}                &       0.2701       &       0.0143       &      18.947     &      0.0000      &       0.2421      &       0.2980       \\
\textbf{prices}                &      -0.5361       &       0.0155       &     -34.507     &      0.0000      &      -0.5665      &      -0.5056       \\
\bottomrule
\end{tabular}
%\caption{IV-2SLS Estimation Summary}
\end{center}

Endogenous: prices \newline
 Instruments: pricestore1, pricestore2, pricestore3, pricestore4, pricestore5, pricestore6, pricestore7, pricestore8, pricestore9, pricestore10, pricestore11, pricestore12, pricestore13, pricestore14, pricestore15, pricestore16, pricestore17, pricestore18, pricestore19, pricestore20, pricestore21, pricestore22, pricestore23, pricestore24, pricestore25, pricestore26, pricestore27, pricestore28, pricestore29, pricestore30 \newline
 Robust Covariance (Heteroskedastic) \newline
 Debiased: False
%%% Local Variables:
%%% mode: latex
%%% TeX-master: t
%%% End:

  \end{table}

\begin{table}[htp]
  \centering
  \begin{center}
\begin{tabular}{lclc}
\toprule
\textbf{Dep. Variable:}                        &         Y          & \textbf{  R-squared:         } &      0.7183      \\
\textbf{Estimator:}                            &      IV-2SLS       & \textbf{  Adj. R-squared:    } &      0.7123      \\
\textbf{No. Observations:}                     &       38544        & \textbf{  F-statistic:       } &    1.711e+05     \\
\textbf{Date:}                                 &  Thu, Nov 04 2021  & \textbf{  P-value (F-stat)   } &      0.0000      \\
\textbf{Time:}                                 &      05:42:30      & \textbf{  Distribution:      } &    chi2(804)     \\
\textbf{Cov. Estimator:}                       &       robust       & \textbf{                     } &                  \\
\textbf{}                                      &                    & \textbf{                     } &                  \\
\bottomrule
\end{tabular}
\begin{tabular}{lcccccc}
                                               & \textbf{Parameter} & \textbf{Std. Err.} & \textbf{T-stat} & \textbf{P-value} & \textbf{Lower CI} & \textbf{Upper CI}  \\
\midrule
\textbf{Intercept}                             &      -5.4606       &       0.0662       &     -82.482     &      0.0000      &      -5.5903      &      -5.3308       \\
\textbf{prom\_}                                &       0.2630       &       0.0130       &      20.297     &      0.0000      &       0.2376      &       0.2884       \\
\textbf{prices}                                &      -0.5454       &       0.0138       &     -39.560     &      0.0000      &      -0.5725      &      -0.5184       \\
\bottomrule
\end{tabular}
%\caption{IV-2SLS Estimation Summary}
\end{center}

Endogenous: prices \newline
 Instruments: pricestore1, pricestore2, pricestore3, pricestore4, pricestore5, pricestore6, pricestore7, pricestore8, pricestore9, pricestore10, pricestore11, pricestore12, pricestore13, pricestore14, pricestore15, pricestore16, pricestore17, pricestore18, pricestore19, pricestore20, pricestore21, pricestore22, pricestore23, pricestore24, pricestore25, pricestore26, pricestore27, pricestore28, pricestore29, pricestore30 \newline
 Robust Covariance (Heteroskedastic) Dummies omitted.\newline
 Debiased: False
%%% Local Variables:
%%% mode: latex
%%% TeX-master: t
%%% End:

  \end{table}

\newpage
\newpage
\subsection*{Mean own-price elasticities from the estimates in models 1--3}

\begin{table}[htp]
  \centering
  \begin{tabular}{lrrrrrrrrr}
\toprule
{} &      OLS1 &      OLS2 &      OLS3 &     IV4.1 &     IV4.2 &     IV4.3 &     IV5.1 &     IV5.2 &     IV5.3 \\
brand &           &           &           &           &           &           &           &           &           \\
\midrule
1     & -0.852929 & -1.166183 & -1.128481 & -0.706043 & -0.027733 & -0.118239 & -0.862496 & -1.832176 & -1.864240 \\
2     & -1.232714 & -1.685451 & -1.630960 & -1.020423 & -0.040082 & -0.170887 & -1.246541 & -2.647991 & -2.694332 \\
3     & -1.750607 & -2.393550 & -2.316167 & -1.449128 & -0.056921 & -0.242681 & -1.770243 & -3.760477 & -3.826287 \\
4     & -0.739114 & -1.010568 & -0.977896 & -0.611828 & -0.024032 & -0.102461 & -0.747405 & -1.587691 & -1.615476 \\
5     & -1.283685 & -1.755141 & -1.698398 & -1.062616 & -0.041739 & -0.177953 & -1.298083 & -2.757481 & -2.805738 \\
6     & -2.036190 & -2.784018 & -2.694011 & -1.685529 & -0.066207 & -0.282271 & -2.059029 & -4.373936 & -4.450483 \\
7     & -0.666923 & -0.911862 & -0.882382 & -0.552069 & -0.021685 & -0.092453 & -0.674403 & -1.432616 & -1.457688 \\
8     & -0.900107 & -1.230688 & -1.190900 & -0.745096 & -0.029267 & -0.124779 & -0.910203 & -1.933518 & -1.967356 \\
9     & -0.989782 & -1.353297 & -1.309545 & -0.819327 & -0.032183 & -0.137210 & -1.000884 & -2.126149 & -2.163357 \\
10    & -0.481135 & -0.657841 & -0.636573 & -0.398277 & -0.015644 & -0.066698 & -0.486532 & -1.033526 & -1.051613 \\
11    & -1.109697 & -1.517254 & -1.468201 & -0.918591 & -0.036082 & -0.153834 & -1.122144 & -2.383739 & -2.425456 \\
\bottomrule
\end{tabular}

%%% Local Variables:
%%% mode: latex
%%% TeX-master: t
%%% End:

  \end{table}

These results make sense. As a rule of thumb, the own-price elasticities should be $\in (-2, -5)$. The IV estimates using the Hausman instrument are approximately in this range. The OLS estimates are not, indicating that endogeneity is a practical concern in this setting. The estimates with wholesale cost as an instrument are also ``too small'', indicating that wholesale cost may not be a viable instrument in this context.

\newpage
\section*{Problem 2}
Our results for this section don't make sense, and there are a few indications that we may have errors in our code. Our GMM function reached the maximum number of iterations. It's not clear if we set the tolerance level too low or if it simply failed to converge; if it didn't converge then the estimation procedure is incorrect and our results \emph{shouldn't} make sense. 
\subsection*{Estimate the parameter values using BLP}
\begin{table}[htp]
  \centering
  \begin{tabular}{ccccc}
    $\alpha$ & $\beta$ & $\sigma_{ib}$ & $\sigma_I$ & $\sigma_I^2$\\
    \hline
    $1.67671625$ & $\begin{bmatrix}
 40.5539681 \\
       -37.20820459\\
        22.10159763\\
        -9.87790967\\
       -18.56618279\\
        17.5039916 \\
       -32.63598696\\
        34.20049778\\
       -12.49605257\\
         0.95162148\\
       -53.64393846\\
    \end{bmatrix}$
             & $\begin{bmatrix}
0.35372042 \\ 0.4527695 \\ 0.2383683
               \end{bmatrix}$ & $0.005161$ & $-0.121263$
  \end{tabular}
\end{table}
Note that we have one more element of $\beta$ than we should. When we constructed the dummies, we neglected to exclude one dummy variable, but we were unable to re-run our code before submitting our results. Given more time, we would re-run this step, and we suspect that the coefficients would then make more sense.

\subsection*{What are the elasticities for store 9 in week 10?}
% We under-estimated the difficulty of this question. We know that we can calculate this from the first order conditions, and that we need to take a numerical integral over consumers, but we don't know how to do this.
We are going to use the approximation
\[
\eta^k_j \approx \frac{p_k}{s_j}\sum_i (\alpha + \sigma_I I_i) (-s_{ij} s_{ik} + \mathbb{1}_{k = j} s_{ik}).
\]
For the purposes of this question $i$ is a singleton and there is no joint ownership.

\begin{table}[htp]
  \centering
    \begin{tabular}{lrrrrrrrrrrr}
\toprule
{} &          0 &          1 &          2 &          3 &          4 &           5 &          6 &          7 &          8 &          9 &         10 \\
\midrule
0  & -39.302792 &  -0.019898 &  -0.006467 &  -0.015919 &  -0.006964 &   -0.000497 &  -0.011939 &  -0.004975 &  -0.001492 &  -0.010944 &  -0.000497 \\
1  &  -0.029454 & -58.177689 &  -0.009573 &  -0.023563 &  -0.010309 &   -0.000736 &  -0.017672 &  -0.007364 &  -0.002209 &  -0.016200 &  -0.000736 \\
2  &  -0.038587 &  -0.038587 & -76.190310 &  -0.030869 &  -0.013505 &   -0.000965 &  -0.023152 &  -0.009647 &  -0.002894 &  -0.021223 &  -0.000965 \\
3  &  -0.017116 &  -0.017116 &  -0.005563 & -33.804145 &  -0.005991 &   -0.000428 &  -0.010270 &  -0.004279 &  -0.001284 &  -0.009414 &  -0.000428 \\
4  &  -0.031994 &  -0.031994 &  -0.010398 &  -0.025595 & -63.174270 &   -0.000800 &  -0.019197 &  -0.007999 &  -0.002400 &  -0.017597 &  -0.000800 \\
5  &  -0.050743 &  -0.050743 &  -0.016492 &  -0.040595 &  -0.017760 & -100.178617 &  -0.030446 &  -0.012686 &  -0.003806 &  -0.027909 &  -0.001269 \\
6  &  -0.016390 &  -0.016390 &  -0.005327 &  -0.013112 &  -0.005737 &   -0.000410 & -32.367476 &  -0.004098 &  -0.001229 &  -0.009015 &  -0.000410 \\
7  &  -0.020201 &  -0.020201 &  -0.006565 &  -0.016160 &  -0.007070 &   -0.000505 &  -0.012120 & -39.884948 &  -0.001515 &  -0.011110 &  -0.000505 \\
8  &  -0.024011 &  -0.024011 &  -0.007804 &  -0.019209 &  -0.008404 &   -0.000600 &  -0.014406 &  -0.006003 & -47.403955 &  -0.013206 &  -0.000600 \\
9  &  -0.010221 &  -0.010221 &  -0.003322 &  -0.008177 &  -0.003577 &   -0.000256 &  -0.006133 &  -0.002555 &  -0.000767 & -20.184372 &  -0.000256 \\
10 &  -0.027156 &  -0.027156 &  -0.008826 &  -0.021725 &  -0.009505 &   -0.000679 &  -0.016293 &  -0.006789 &  -0.002037 &  -0.014936 & -53.611680 \\
\bottomrule
\end{tabular}

%%% Local Variables:
%%% mode: latex
%%% TeX-master: "ackerman_pset1"
%%% End:

\end{table}

These elasticities are not believable. The important/theoretical difference from the logit elasticities is that elasticities are no longer simply a function of shares, and are now a function of product characteristics as well (so for instance consumers are more likely to switch from one branded product to another than from a branded product to a generic, even if the branded product and generic have similar market shares). The striking problems are the massive magnitudes for own-price elasticities and the uniform negativity of the cross-price elasticities. This is most likely due to an error in our parameter estimates from the previous section.

\subsection*{Back out the marginal costs for store 9 in week 10. How are they different from wholesale costs?}

Since we have a single-ownership structure, we can use scalars everywhere instead of matrices. The expression for marginal cost is then
\[
mc = \frac{1}{\eta_k} \cdot \left(\frac{s_k}{p_k}\right) + p_k.
\]

The marginal costs that we estimate for these firms are
\begin{table}[htp]
  \centering
  \begin{tabular}{lr}
\toprule
{Firm} &         Marginal Cost\\
\midrule
0  &  3.289996 \\
1  &  4.869998 \\
2  &  6.380000 \\
3  &  2.829996 \\
4  &  5.289999 \\
5  &  8.390000 \\
6  &  2.709997 \\
7  &  3.339999 \\
8  &  3.970000 \\
9  &  1.689992 \\
10 &  4.490000 \\
\bottomrule
\end{tabular}

%%% Local Variables:
%%% mode: latex
%%% TeX-master: "ackerman_pset1"
%%% End:

\end{table}

These marginal costs are well above the wholesale costs. Because of our incredibly large elasticity estimates, the implied markups in our model are very very small, so $mc \approx p$. However we see that wholesale cost is notably cheaper than actual prices.

\newpage
\section*{Problem 3}
\subsection*{Predict the post-merger prices using the logit model, but only for store 9 in week 10}

\begin{table}[htp]
  \centering
  \begin{tabular}{lrrrrrr}
\toprule
{} &  store &  week &  brand &  prices &  new\_prices &  price\_change \\
\midrule
33 &      9 &    10 &      1 &    3.29 &    3.235631 &     -0.054369 \\
34 &      9 &    10 &      2 &    4.87 &    4.815631 &     -0.054369 \\
35 &      9 &    10 &      3 &    6.38 &    6.325631 &     -0.054369 \\
36 &      9 &    10 &      4 &    2.83 &    2.797369 &     -0.032631 \\
37 &      9 &    10 &      5 &    5.29 &    5.257369 &     -0.032631 \\
38 &      9 &    10 &      6 &    8.39 &    8.357369 &     -0.032631 \\
39 &      9 &    10 &      7 &    2.71 &    2.694294 &     -0.015706 \\
40 &      9 &    10 &      8 &    3.34 &    3.324294 &     -0.015706 \\
41 &      9 &    10 &      9 &    3.97 &    3.954294 &     -0.015706 \\
42 &      9 &    10 &     10 &    1.69 &    1.671601 &     -0.018399 \\
43 &      9 &    10 &     11 &    4.49 &    4.471601 &     -0.018399 \\
\bottomrule
\end{tabular}

%%% Local Variables:
%%% mode: latex
%%% TeX-master: "ackerman_pset1"
%%% End:

\end{table}

\subsection*{How to predict the change in prices after the merger using the random coefficients model?}

The procedure for predicting the effects of a merger using estimates from the random coefficients model is exactly the same. When we estimate the logit model, we end up with unrealistic elasticities, and hence unrealistic substitution patterns. The random coefficients model gives us a more reasonable matrix of elasticities, but we use it in the same way. When we want to assess the change of moving from two single-product firms to a multi-product firm, we switch from looking at firms that take FOCs with respect to a single price to a single firm that takes FOCs with respect to two prices (the price of each good that it produces). What does this mean? With a more realistic estimate of substitution patterns, we have a better idea of how consumers will respond to changes in prices. Pre-merger, a firm would be ``unwilling'' to raise its price because it would lose customers. Now, however, the merged firm \emph{may} recognize that increasing the price of one of it's products will cause consumers to switch to its other product, so its profit maximizing price may be higher in the merged case than in the pre-merger case. The random coefficients model is giving us a better insight into these actual substitution patterns.


\newpage
\inputminted{python}{blp_hw.py}
\end{document}
%%% Local Variables:
%%% mode: latex
%%% TeX-master: t
%%% End:
