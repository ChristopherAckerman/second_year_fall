\input{../article_preamble}
\newtheorem*{definition}{Definition}
\title{Networks Pset \#1}
\date{\today}
\begin{document}
\author{Chris Ackerman\thanks{I worked on this problem set with Aristotle Magganas, Antonio Martner and Koki Okumura.}}

\maketitle
\section*{Question 1}
\begin{align*}
  p(n) &\equiv \frac{\lambda}{n}, \quad \lambda \in (0, \infty)\\
  \intertext{The degree distribution is Poisson, }
  p_k &= \frac{\lambda^k}{k!} e^{- \lambda}\\
  p &\equiv 1 - q\\
  \therefore p&= \sum^\infty_{k = 0} p_k p^k\\
  \iff 1 - q &= \sum^\infty_{k = 0} \frac{\lambda^k}{k!}e^{-\lambda} (1 - q)^k \tag{1}\\
  \sum^\infty_{k = 0} \frac{(\lambda (1 - q))^k}{k!} &= e^{\lambda(1 - q)}\\
  \intertext{so the labeled equation becomes}
  1 - q &= e^{-\lambda q}\\
  \iff q &= 1 - e^{-\lambda q}
\end{align*}
If $\lambda <1$, then the average degree is less than one and there is no giant component ($q = 0$). Then the extinction probability is $1$. If instead the average degree is greater than $1$, there is a giant component $q \in (0, 1)$ and the extinction probability is $p \in (0, 1) < 1 $.
\section*{Question 3}
Fix $p \in (0, 1)$ and some network $q$ on $k$ nodes. Now consider the sequence of Erdos-Renyi networks $G(n, p)$. Partition the $n$ nodes into as many separable groups of $k$ nodes as possible, and consider the subnetworks that form in each group. Let $\ell$ be the number of links in network $g$. The probability that none of these subnetworks matches the network $g$ is
\[
\left(1 - p^\ell (1- p )^{\frac{k(k - 1)}{2} - \ell}\right)^n \underset{n \to \infty}{\to} 0
\]
Thus the probability that there is a copy of the random network goes to $1$ as $n \to \infty$.
\section*{Question 4}
\begin{definition}[Connected Network]
 For any two nodes in the network, there exists a path between them. 
\end{definition}
\begin{definition}[Disconnected Network]
A network with at least one vertex that does not have a path.
\end{definition}
Suppose $g$ is a disconnected network. Take $a, b \in g$ and suppose they are in different components. Then, $\not \exists (a, b) \in g$. By the definition of the complement of a network, $(a, b)$ must be an edge in $g^\prime$. Thus there is a path between $a$ and $b$. 


Suppose instead that $(a, b) \in g$. Then $a, b$ are in the same component of $g$. Since $g$ is disconnected, $\exists c$ s.t. $(a, c) \not \in g \land (b, c) \not \in g$. Therefore, $(a, c, b)$ must bea  path from node $a$ to $b$ in $g^\prime$. 
\section*{Question 8}
For even $n$, the efficient network is a collection of dyads $ij$. But it is not pairwise stable.


\begin{enumerate}
\item First we will check efficiency. This is straightforward (the question isn't asking us to prove uniqueness, just to show that the dyad network \emph{is} efficient, so step 1 is to maximize social welfare, and then step 2 is to show that the dyad network achieves this number).
  \begin{definition}[Efficiency, Jackson p.15]
    A network $g$ is \gr{efficient} if it maximizes $\sum_i u_i (g)$.
  \end{definition}
  The maximal possible utilitarian welfare in the coauthor model is
  \begin{align*}
    \sum_{i \in N} u_i(g) &= \sum_{i:d_i(g) > 0} \sum_{j: d_j (g) > 0} \left[\frac{1}{d_i(g)} + \frac{1}{d_j(g)} + \frac{1}{d_i(g) d_j(g)}\right]\\
    &\le 2N + \sum_{i:d_i(g) > 0} \sum_{j: d_j (g) > 0} + \frac{1}{d_i(g) d_j(g)}\\
    \intertext{Since we're only summing over nodes with at least one connection,}
    \frac{1}{d_i(g) d_j(g)} & \le \frac{1}{1 \cdot 1}\\
    \implies \sum_{i:d_i(g) > 0} \sum_{j: d_j (g) > 0} + \frac{1}{d_i(g) d_j(g)} &\le N\\
    \implies \sum_{i \in N} u_i(g) &\le 3N\\
    \intertext{Now, we just need to show that the dyad network achieves this upper bound. Note that $d_i = d_j = 1$ for every node in the dyad network, so}
    \sum_{i \in N} u_i(g) &= \sum_{i:d_i(g) > 0} \sum_{j: d_j (g) > 0} \left[\frac{1}{d_i(g)} + \frac{1}{d_j(g)} + \frac{1}{d_i(g) d_j(g)}\right]\\
                          &= \sum_i \sum_j \frac{1}{1} + \frac{1}{1} + \frac{1}{1}\\
    &= 3N \qed
  \end{align*}
  \item Now pairwise stability. The procedure is similar; check the definition and then show that each individual in the dyad network wants to deviate (add at least 1 more connection).
    \begin{definition}[Pairwise Stability, Jackson p. 16]
      $ $
      \begin{enumerate}
      \item No agent can increase his or her utility by deleting a link that he or she is directly involved in.
        \item No two agents can both benefit (at least one strictly) by adding a link between themselves.
      \end{enumerate}
    \end{definition}
    We're going to find a counterexample to $(b)$ by looking at two agents who link together, so that each of them have two coauthors, and one of their coauthors has 1 coauthor, while the other (the guy deviating) has two coauthors. The payoffs should be strictly higher for both players, since everything is symmetric. Define $g$ as the dyad network and $g^\prime$ is the network where two authors have chosen to write another paper together. We're going to look at $u_i$, the utility of one of the guys who deviates.
    \begin{align*}
      u_i(g) &= \sum_{j: ij \in g} \left(\frac{1}{d_i(g)} + \frac{1}{d_j (g)} + \frac{1}{d_i(g) d_j(g)}\right)\\
             &= 1 + 1 + 1\\
      &= 3\\
      u_i(g^\prime) &= \sum_{j: ij \in g^\prime} \left(\frac{1}{d_i(g)} + \frac{1}{d_j (g)} + \frac{1}{d_i(g) d_j(g)}\right)\\
             &= \frac{1}{2} + \frac{1}{1} + \frac{1}{2}\\
             &\quad + \frac{1}{2} + \frac{1}{2} + \frac{1}{4}\\
             &= \frac{13}{4}\\
      &> u_i(g) \Rightarrow\Leftarrow
    \end{align*}
\end{enumerate}
\end{document}
%%% Local Variables:
%%% mode: latex
%%% TeX-master: t
%%% End:
