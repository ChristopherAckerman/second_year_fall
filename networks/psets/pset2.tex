\documentclass[dvipsnames]{article}
\usepackage{amsmath,amsthm,amssymb}
\usepackage{graphicx}
\usepackage{hyperref}
\usepackage{textcomp}
% \usepackage{dsfont}
\usepackage{tabularx}
\usepackage{tikz}
\usepackage{physics}
\usepackage{changepage}% http://ctan.org/pkg/changepage
\usetikzlibrary{scopes,calc,arrows}
\usepackage{setspace}
\usepackage[makeroom]{cancel}
\usepackage{enumitem}
\usepackage[margin=1in]{geometry}
\usepackage[T1]{fontenc}
\usepackage[utf8]{inputenc}
\usepackage{tabularx,ragged2e,booktabs,caption}
\usepackage{wrapfig,lipsum,booktabs}
\usepackage{hanging}
\usepackage{multicol}
\usepackage{multirow}
\usepackage{blindtext}
\usepackage{booktabs}
\usepackage{color}
\usepackage{dcolumn}
% \usepackage{minted}
% \definecolor{light}{rgb}{0.35, 0.35, 0.35}
% \def\light#1{{\color{light}#1}}
\usepackage{mathtools}
\newcommand{\p}{\mathbb{P}}
\newcommand{\E}{\mathbb{E}}
\newcommand{\R}{\mathbb{R}}
\newcommand{\Var}{\operatorname{Var}}
\newcommand{\gr}{\textcolor{ForestGreen}}
\newcommand{\rd}{\textcolor{red}}


\newtheorem*{definition}{Definition}
\title{Networks Pset \#1}
\date{\today}
\begin{document}
\author{Chris Ackerman\thanks{I worked on this problem set with Aristotle Magganas, Antonio Martner and Koki Okumura.}}

\maketitle

\section*{Question 3}

If we make the two leftmost red nodes green, the resulting green nodes are $\frac{2}{3}$-cohesive, and the remaining red nodes are $\frac{3}{4}$-cohesive. The original green nodes only have green neighbors, and the two new ones each have $2$ green neighbors and a single red neighbor, so the green subnetwork is $\frac{2}{3}$-cohesive. Two red nodes are connected to a green node, but they are each connected to three red nodes, so the new red subnetwork is $\frac{3}{4}$-cohesive.

\section*{Question 4}
\begin{enumerate}[label=\alph*)]
\item The highest $q^\ast(g)$ we can achieve is $\frac{1}{2}$, and the line and circle networks will both achieve $q^\ast(g) = \frac{1}{2}$.
  \item The lowest $q^\ast(g)$ we can achieve is $\frac{1}{9}$, and the complete graph achieves this $q^\ast(g)$.
\end{enumerate}

\section*{Question 5}
Without a network, the monopolist solves the problem
\begin{align*}
  \max_{p} &\ p \cdot q \tag{Assume $mc = 0$}\\
  \theta \sim U[0, 1] &\implies q  = 1 - p\\
  \max_p &\ p \cdot (1 - p)\\
  \frac{\partial}{\partial p} p - p^2 &= 0 \tag{FOC}\\
  1 - 2p &= 0\\
  p &= \frac{1}{2}
\end{align*}
Now let's put this on a network. We just need to cook up a counterexample, so here's a funny network that will work. Put all $\theta < \frac{1}{2} + \delta, \delta > 0$ into singletons. Put all $\theta \ge \frac{1}{2} + \delta$ in a completely connected network. I guess the best economic ``story'' for this is that all of the rich guys live in a gated community and hang out together, and all the poor guys are on their own somewhere else. So not \emph{too} crazy. Now, as we send $\varepsilon \to 0$ we only hit the giant component (all of the $\theta \in \left[\frac{1}{2} + \delta, 1\right]$). Let's see how the monopolist's price without a network does, and if we can do better.
\begin{align*}
  \pi &= p \cdot q\\
      &= p \cdot \left(1 - \left[\frac{1}{2} + \delta\right]\right)\\
  &= \left(\frac{1}{2}\right)^2 - \frac{\delta}{2}
\end{align*}
But the monopolist can do better! $\delta > 0 \implies \theta > p $ for all the guys that are actually buying this product. Everyone is getting some consumer surplus. We can squeeze \emph{all} of this surplus out of the guy with $\underline{\theta} = \frac{1}{2} + \delta$, and we can squeeze \emph{some} surplus ($\delta$) out of everyone that's buying. Basically, increasing the price to $p^\prime \in \left(\frac{1}{2}, \frac{1}{2} + \delta\right]$ doesn't cause anybody to stop buying the product, but does give a higher per-unit price to the monopolist.
\begin{align*}
  \pi^\prime &= p^\prime \cdot q\\
             &= \underbrace{\left(\frac{1}{2} + \delta\right)}_{\text{higher price}} \underbrace{\left(1 - \left[\frac{1}{2} + \delta\right]\right)}_{\text{same quantity}}\\
             &= \left(\frac{1}{2}\right)^2 - \frac{\delta}{2} + \underbrace{\delta \left(\frac{1}{2} - \delta\right)}_{\text{additional profit}}\\
  &> \pi
\end{align*}
This holds for all $\delta \in \left(0, \frac{1}{2}\right)$.

\section*{Question 7}
Proof by contradiction and induction. Choose agent $i_1$ in a finite network, and suppose he is \emph{not} in a strongly connected or closed group, and does \emph{not} have a directed path to an agent in a strongly connected group. Then, there exist two nodes $i_2$ and $i_3$ such that
\begin{enumerate}
\item there is a path from $i_1 \to i_2$
\item $i_2$ is not in a strongly connected and closed group
\item there is a path from $i_2 \to i_3$
  \item $i_3$ is not in a strongly connected and closed group
\end{enumerate}
By induction we can keep going, but then the network isn't finite. Therefore, agent $i_1$ is either in a strongly connected and closed group or has a directed path to an agent in a strongly connected and closed group. And if the agent is either in a strongly connected and closed group or has a directed path into one, there must be at least one strongly connected and closed group.
\end{document}
%%% Local Variables:
%%% mode: latex
%%% TeX-master: t
%%% End:
