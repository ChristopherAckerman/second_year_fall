\input{../article_preamble.tex}
\title{Notes for \emph{The Network Origins of Aggregate Fluctuations}}
\author{Chris Ackerman}
\date{\today}


\begin{document}
\maketitle
\section{Overview}
\begin{itemize}
\item Paper was written shortly after the 2008 Financial Crisis
\item \textbf{Leading question:} How does the organization of the input-output network in an economy affect economic volatility?
\item \textbf{Leading example: One sector}
  \begin{itemize}
  \item If there's one sector and one firm, then any shock to that firm shuts down the entire economy.
    \item But if we send $n_{\text{firms}} \to \infty$ then shocks to individual firms become unimportant, and the economy is resilient to uncorrelated shocks.
    \end{itemize}
  \item \textbf{Leading Counterexample: Automakers during 2008}
    \begin{itemize}
    \item Some suppliers make small but important parts for all the carmakers
    \item If one of these small companies collapses, there's a chance that none of the carmakers will be able to continue making cars
      \item Ford asked congress to support Chrysler and GM so that their mutual suppliers wouldn't experience any disruptions
    \end{itemize}
    \item \textbf{Research Question: } Can we write down a network model of these interactions to generalize this insight? And can we fit this model to actual input/output data from the US?
      \item \rd{NB we can put figures 1--3 on slides to show the 2 leading examples of symmetric networks, and then the actual 1997 network to show that it exhibits meaningful asymmetry.}
\end{itemize}

\section{Approach}
\begin{itemize}
\item Consider a sequence of economies $\{\mathcal{E}_n\}_{n \in \mathbb{N}}$ corresponding to different levels of disaggregation
\item Each economy $\mathcal{E}_n$ has $n$ sectors whose input requirements are captured by an $n \times n$ matrix $W_n$
  \item entry $(i, j)$ captures the share of sector $j$'s product in sector $i$'s production technology
  \item $j$th column sum, the \emph{degree} of sector $j$, is the share of $j$'s output in the input of the entire economy
  \item Given a sequence of economies $\{\mathcal{E}_n\}_{n \in \mathbb{N}}$, investigate whether \emph{aggregate volatility} (st. dev. of log otuput) vanishes as $n \to \infty$.
    \begin{itemize}
    \item \textbf{Preview of results: SOMETIMES}
    \end{itemize}
  \item Main focus: when does LLN hold and the network structure has an important effect on aggregate fluctuations
    \begin{itemize}
    \item Aggregate output might concentrate around its mean at a rate slower than $\sqrt{n}$; sectoral shocks may have a significant role in creating aggregate shocks, even if a disaggregated economy
    \item Two causes of slow rates of aggregate volatility decay:
      \begin{enumerate}
      \item First-order interconnections: shocks to a sector that is a supplier to lots of other sectors; direct propagation
        \item Higher-order interconnections: low productivity in one sector might reduce productivity in a sequence of interconnected sectors
      \end{enumerate}
    \end{itemize}
\end{itemize}

\section{Results}
\begin{itemize}
\item[Theorem 2:] Provides a lower bound on asymmetry across sectors captured by variation in sectoral degrees. Higher variation in the degree of different sectors implies lower rates of decay for aggregate volatility.
  \item[Theorem 3:] Tighter lower bound on second-order interconnectivity between different sectors. Two economics with identical empirical degree distributions (first-order connections) may have significantly different levels of aggregate volatility because of interactions with downstream sectors.
    \item[Theorem 4:] Sectoral shocks average out at rate $\sqrt{n}$ for \emph{balanced} networks. The nature of aggregate fluctuations resulting from sectoral shocks is not related to the sparseness of the input-output matrix, but the extent of asymmetry between sectors.
    \item Empirical exercise (section 4):
      \begin{itemize}
      \item Empirical distribution of both first- and second-order degrees have Pareto tails, with second-order tail having a shape parameter of $\zeta = 1.18$
      \item If this degree distribution also holds for large $n$, aggregate volatility in the US economy decays at rate slower than $n^{0.15}$
      \item US input-output network more similar to a star network than complete network
        \item In practice we might see sizable aggregate fluctuations from idiosyncratic shocks to different sectors
      \end{itemize}
\end{itemize}

\section{Model}
\begin{itemize}
\item Representative household with one unit of inelastic labor; Cobb-Douglas preferences over $n$ distinct goods:
  \[
u(c_1, c_2,\ldots, c_n) = A \prod^n_{i = 1} (c_i)^{1/n}
  \]
\item Each good is produced by a competitive sector and can be either consumed or used by other sectors as an input for production. Each sector uses C-D production with CRS; output of sector $i$ is
  \[
x_{i}=z_{i}^{\alpha} \ell_{i}^{\alpha} \prod_{j=1}^{n} x_{i j}^{(1-\alpha) w_{i j}}
  \]
  $z_i$ are productivity shocks across sectors, $\varepsilon_i \equiv \log(z_i) \sim F_i$. $w_{ij} \ge 0$ is the share of good $j$ in the intermediate inputs of firms in sector $i$. This definition is nice because $w_{ij}$ corresponds to the entries of input-output tables; will use this in section 4 for the calibration bit.
\item Can summarize the structure of intersectoral trade with the input-output matrix $W$, which has entries $w_{ij}$.
\item Economy is completely specified by the tuple
  \[
\mathcal{E} = (\mathcal{I}, W, \{F_i\}_{i \in \mathcal{I}}), 
\]
where $\mathcal{I}$ is the number of sectors.
\item Can equivalently represent the economy as a weighted directed graph on $n$ vertices
\end{itemize}
\end{document}

%%% Local Variables:
%%% mode: latex
%%% TeX-master: t
%%% End:
